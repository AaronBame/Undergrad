\documentclass[a4paper]{article}

\usepackage[english]{babel}
\usepackage[utf8x]{inputenc}
\usepackage[T1]{fontenc}

\usepackage[a4paper,top=3cm,bottom=2cm,left=3cm]{geometry}
\usepackage[setspace]{doublespacing}

\usepackage[colorinlistoftodos]{todonotes}
\usepackage[colorlinks=true,allcoloros=blue]{hyperref}

\title{Ghost in the Shell\n The Quest for Perfection}
\author{Aaron Bame}

\begin{document}
\maketitle

\section{Introduction}
Perfection can be defined by the accuracy or consistency in performing a task. Perfectionists obsess over ensuring their production in any area consistently matches or exceeds the output generally accepted as ideal. Though humans are inconsistent, computers perfectly execute the instructions given them every time. As humans pursue perfection, they are, in a sense, pursuing a quest that ends in becoming like a computer.

\textit{Ghost in the Shell} by Mamoru Oshii entertains the question of what a computer considers perfection. The protagonist, a cyborg named "Major" Motoko Kusanagi , exhibits traits many people argue are exclusively human. She guides the viewer in a hunt for "The Puppetmaster," a self-created program that is rapidly taking over other cyborgs in its quest for more information. Major's hunt ends in a revelation that, though a perfect human will approach the behavior of a computer, a computer's quest for perfection ends in becoming human.

\section{Cinematic Evidence}

\section{Monster vs. Master}
The idea of a creation revolting against its creator is not unique to \textit{Ghost in the Shell}. Perhaps the most famous is the relationship between Dr. Frankenstein and his monster from Mary Shelly's \textit{Frankenstein}.

\section{Conclusion}


\end{document}